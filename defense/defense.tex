\documentclass{beamer}


\usepackage[utf8]{inputenc}
\usepackage{amsmath}
\usepackage{amsfonts}
\usepackage{amssymb}
\usepackage{graphicx}
\usepackage{ragged2e}  % `\justifying` text
\usepackage{booktabs}  % Tables
\usepackage{tabularx}
\usepackage{tikz}      % Diagrams
\usetikzlibrary{calc, shapes, backgrounds}
\usepackage{amsmath}
\usepackage{amssymb}
\usepackage{dsfont}
\usepackage{url}       % `\url
\usepackage{listings}  % Code listings
\usepackage[T1]{fontenc}


\usepackage{theme/beamerthemehbrs}

\author[Ethan Oswald Massey]{Ethan Oswald Massey}
\title{Comparative Analysis of Techniques for Spatio-Temporal World Modeling}
%\subtitle{Subtitle of presentation}
%\title{Techniques for Spatio-Temporal World Modeling}
%\subtitle{A Comparative Analysis}
\institute[HBRS]{Hochschule Bonn-Rhein-Sieg}
\date{2019-04-16}
%\subject{Test beamer}

% \thirdpartylogo{path/to/your/image}


\begin{document}
{
\begin{frame}
\titlepage
\end{frame}
}

\begin{frame}[t]{Breakdown of Research Topic}
\begin{itemize}
    \setlength\itemsep{1em}

    \item \textbf{World Modeling}
    \begin{itemize}
      \item “an internal representation of their [a robot's] surroundings to be able to reason
        about and to interact with it”\cite{krajnik2015}
    \item Historically, only two or three dimensional
    \end{itemize}

    \item \textbf{Spatio-Temporal}
    \begin{itemize}
      \item Introduction of the fourth dimension
      \item Allows for prediction of changes in environment
    \end{itemize}

    \item \textbf{Comparative Analysis}
    \begin{itemize}
      \item Various techniques exist and are still emerging
      \item How can someone choose between these methods?
      \item Establish criteria for comparison and evaluation
    \end{itemize}

\end{itemize}
\end{frame}


\begin{frame}[t]{Uses of Spatio-Temporal World Models}
  \begin{itemize}
    \setlength\itemsep{1em}
    \item Useful where real-world changes are predictable and frequent
    \item Human environments are often quite predictable when generalized
    \item Logistics network (e.g. ROPOD)
  \end{itemize}

  \begin{block}{Common Predictions}
    \begin{itemize}
    \item Availability of resources
      \begin{itemize}
        \item Doors open/closed
        \item How long until the next elevator
      \end{itemize}
    \item Human traffic density
    \end{itemize}
  \end{block}
\end{frame}


\begin{frame}{Benefits of Selecting the Best Model}
  \centering
    Improved predictions
    \\
    $\downarrow$
    \\
    Improves path planning
    \\
    $\downarrow$
    \\
    Decreases travel time
    \\
    $\downarrow$
    \\
    Increases work throughput
    \\
    $\downarrow$
    \\
    Saves time \& money
    \\
\end{frame}


\begin{frame}[t]{Previous Work}

  \begin{itemize}
    \item \scriptsize Occupancy Grid Models for Robot Mapping in Changing Environments (D. Meyer-Delius et al. 2012) \cite{Meyer-Delius2012} \normalsize
    \begin{itemize}
      \item Internal comparison based on prediction accuracy
    \end{itemize}

    \item \scriptsize FreMEn: Frequency Map Enhancement for Long-Term Mobile Robot Autonomy in Changing Environments \cite{krajnik2015} \normalsize
    \begin{itemize}
      \item Basic discussion about space and time complexity
    \end{itemize}

  \item \scriptsize A Poisson-Spectral Model for Modelling Temporal Patterns in Human Data Observed by a Robot \cite{Jovan2016} \normalsize
    \begin{itemize}
      \item Discussion of prediction accuracy and representation
    \end{itemize}
  \end{itemize}

  \begin{block}{Limitations}
    \begin{itemize}
      \item The majority of these approaches are inward focused
      \item Little to no comparison with other existing methods
      \item Limited in scope and depth
    \end{itemize}
  \end{block}

\end{frame}


\begin{frame}[t]{Establishing Criteria for Comparison}
  \framesubtitle{Characteristic Information}
  \begin{itemize}
    \setlength\itemsep{1em}
    \item Computational Complexity
      \begin{itemize}
        \item Theoretical space and time complexity (Big O)
      \end{itemize}

    \item Efficiency of Data Storage
      \begin{itemize}
        \item Space taken to store observations and corrosponding fidelity
      \end{itemize}

    \item Learning Method
      \begin{itemize}
        \item Online vs Offline
      \end{itemize}

    \item Map Type Dependency
      \begin{itemize}
        \item e.g. methods only works with 2D occupancy girds
      \end{itemize}

    \item Limitations on Predictions
      \begin{itemize}
        \item Classification restrictions (e.g. binary vs non-binary)
        \item Historical and longterm future predictions
      \end{itemize}

  \end{itemize}
\end{frame}

\begin{frame}[t]{Establishing Criteria for Comparison}
  \framesubtitle{Metainformation}
  \begin{itemize}
    \setlength\itemsep{1em}
    \item Availability of Work (1-3)
      \begin{itemize}
        \item 1 - Freely available
        \item 2 - Some, but not all, components available
        \item 3 - Limited or no resources
      \end{itemize}

    \item Implementation Complexity (1-3)
      \begin{itemize}
        \item 1 - Combination of preexisting models or mathematical constructs
        \item 2 - Multiple complex components
        \item 3 - Cutting edge complex and/or novel approach with multiple components
      \end{itemize}

    \item Suitable Fields of Application
      \begin{itemize}
        \item Generic vs field specific implementations
      \end{itemize}

  \end{itemize}
\end{frame}


\begin{frame}[t]{ROPOD - A Case Study}
    %\framesubtitle{ROPOD}
  \begin{columns}[t]

    \column{.5\textwidth}
    \begin{block}{What is ROPOD?}
      \begin{itemize}
        \item Aims to handle legacy in logistics
        \item Low cost multi-robot solution
        \item Moves supplies \& beds
        \item Operates in a preexisting human environment (hospital)
      \end{itemize}
    \end{block}

    \column{.5\textwidth}
    \begin{block}{Why ROPOD?}
      \begin{itemize}
        \item Real world context
        \item Extensive data collection
        \item Periodic \& dynamic behaviors
        \item Binary and non-binary behaviors
      \end{itemize}
    \end{block}
  \end{columns}
\end{frame}




\begin{frame}[t]{Contributions}

  \begin{itemize}
    \setlength\itemsep{1em}
    \item Completed survey of spatio-temporal world modeling techniques
    \item Defined qualitative criteria and evaluated existing methods
    \item Defined and preformed experiments for benchmarking methods
    \item Applied results of benchmarking to make real-world recommendations
  \end{itemize}

\end{frame}



\begin{frame}[label=bibliography]{Bibliography}
  \begin{thebibliography}{9}
    \bibitem{krajnik2015}
        \emph{FreMEn: Frequency Map Enhancement for Long-Term Mobile Robot Autonomy in Changing Environments}
        T. Krajník, J. Fentanes, J. Santos, \& T. Duckett,
        ICRA Workshop on Visual Place Recognition in Changing Environments
        2015
    \bibitem{Meyer-Delius2012}
        \emph{Occupancy Grid Models for Robot Mapping in Changing Environments},
        Meyer-Delius, Daniel and Beinhofer, Maximilian and Burgard, Wolfram,
        Proc. of the Twenty-Sixth AAAI Conference on Artificial Intelligence
        2012
    \bibitem{Jovan2016}
        \emph{A poisson-spectral model for modelling temporal patterns in human data observed by a robot},
        Jovan, Ferdian and Wyatt, Jeremy and Hawes, Nick and Krajnik, Tomas,
        IEEE International Conference on Intelligent Robots and Systems
        2016
  \end{thebibliography}
\end{frame}




\end{document}
