%!TEX root = ../report.tex

\begin{document}
  \chapter{ Introduction }

  A robot's world model is its internal representation of the environment that
  allows it to reason and make decisions. This ability to make decisions, and
  thus how well a robot performs in an environment, can be directly correlated
  to the quality of the world model that a robot contains. Historically, these models
  have been static 2 or three-dimensional representations, but within the past
  2 decades multiple methods have been developed to introduce an additional
  dimension to these maps, the dimension of time. The inclusion of time allows
  for a robot to make decisions about when it may want to accomplish a task, or
  to avoid a certain area at a given time. In the simple case of an
  office, a robot may learn to avoid areas of high traffic around the cafeteria
  during lunch hours or learn that a shortcut between two buildings is open, but only
  during work hours. These simple examples illustrate the type of knowledge and
  efficiency that can be gleaned by introducing a temporal component to a robot's
  world model. This new type of world model has come to be known as a
  ``spatio-temporal world model'', as it is a model of the world that contains
  spatial information, both the physical environment, and temporal
  information, that is, how the environment changes through time.\\

  Automatically guided vehicles (AGVs), now common in the field of logistics,
  stand to benefit a great deal from these improvements in world modeling.
  Moving from point A to point B is an extremely common task in a wide
  variety of domains including industrial, commercial, and residential
  applications. AGVs have most prominently been used in industrial settings thus
  far, but have been relegated to a discrete and limited set
  of predictable tasks. This is especially true when logistics involve a
  particularly dynamic or human environment. \\

  Recent work into introducing a temporal component to world models has
  already begun and shows great promise. A variety of methods have been
  introduced to allow for an AGV to observe and make predictions about its
  environment through time. However, since this field is relatively new, and
  with new advancements and approaches being introduced yearly, it is becoming
  increasingly difficult to evaluate or choose between the
  different spatio-temporal world modeling options. It is for this reason that a
  method, or set of criteria, be devised for comparing and contrasting the
  variety of solutions. The analysis of these methods will not only allow for
  others to choose the most fitting approach for a given environment, but also
  expose deficiencies in the current approaches and guide future research
  efforts. Improvements in this field will ultimately result in more flexible
  AGVs that can operate in a wider variety of environments and for longer
  periods of autonomy. \\

  \section{ Challenges and Difficulties }

  Historically, world modeling techniques can be thought of as simply a mapping
  and path planning problem in either two-dimensional or three-dimensional space.
  These problems have been studied for decades and thus there already exist a
  handful of well-known solutions, each with its own advantages and
  disadvantages. However, with the fairly recent inclusion of the fourth
  dimension, time, into the equation there has been the introduction of a number
  of different methods. \\

  The early and simplistic approaches to introducing temporal components into
  world models started as early as 2002 \cite{Arbuckle2002} but within the past
  two decades there has been an uptick both in the number of different
  approaches and the complexity of the methods.
  With this increase in complexity and variety of approaches,
  combined with a lack of historical perspective and analysis, it can be
  daunting task to select the ``correct`` let alone a well-fitting spatio-temporal
  world model for a new project. A few papers have introduced some simple means
  for comparison, but these have been limited to a rudimentary discussion about the
  space and time complexity of an approach, or an internal reflection on
  variations of a proposed method. \\


  \section{ Motivation }
  With many different methods and no historical knowledge or criteria of comparison
  this paper aims to provide a template for comparing existing models that should
  be extensible to account for the inevitable release of future methods. To that
  aim the following goals shall be met:

  \begin{itemize}
    \item Summary of the major existing spatio-temporal world modeling techniques.

    \item Collection of performance measurement or other comparison techniques as
      defined by the papers themselves.

    \item Introduction of metainformation in order to better compare the existing
      world modeling techniques

    \item A quick and easy-to-use table for high level overview and comparison of techniques

    \item Example application of the aforementioned information to select a fitting
      technique for a real-world application

    \item Subsequent evaluation and discussion on the appropriateness of technique
      selected, especially with respect to the real-world use and introduced metainformation

  \end{itemize}

  It is with this collection of existing comparison techniques, new metainformation,
  and a tangible example that other projects may be able to more easily evaluate
  and select the best fitting spatio-temporal world modeling technique for the
  project. Additionally, when new spatio-temporal world model techniques are
  introduced, it should be with relative ease that their information be integrated
  into this method for comparative analysis for future use. \\



  \section{ Problem Formulation }

  In order to best choose between preexisting solutions for spatio-temporal
  world modeling and guide future development, it is vital that comparative criteria be
  established. This comparative analysis will set out to clarify and quantify
  these approaches. Although the comparative analysis will be general enough to be
  applicable to any project seeking to incorporate spatio-temporal wold modeling,
  it will be viewed through the lens of a specific real-world application
  with a focus on long-term planning. Additional details about the specific application
  will be discussed later, but it is important to note that viewing the
  various modeling methods through this lens of a real-world application is quite
  powerful. \\

  Improvements in world modeling, specifically within the domain of
  spatio-temporal world modeling, have already yielded significant increases
  in the performance of robotic logistic systems. These improvements led to
  decreases in travel time as well as increases in reliability, which hinge
  on knowing what areas to avoid at which times. In turn, these improvements
  create a much more powerful and scaleable logistics network with less
  downtime. This results in more goods being delivered, and saves
  time, money, and in the case of hospitals, potentially lives. \\

  Despite these benefits, and the number of existing approaches
  for spatio-temporal world modeling, there is currently no method for
  accurately comparing and contrasting multiple approaches for a given task. It is with this
  in mind that this thesis will collect, describe, compare, and contrast these
  approaches. It will use criteria already available, when possible, as some
  of the work already includes basic performance statistics.
  Furthermore, in work where these criteria are not mentioned explicitly, or are
  not otherwise available, an attempt to derive the information will be preformed either via
  calculation or collected via simulation. Lastly, new criteria will be devised
  or otherwise assigned to account for information desired and not provided
  or other metainformation that would aid in comparing these methods. \\

  Finally, an example study will be included that will attempt to determine the
  best-suited approach for a real-world scenario, known as ROPOD. This
  scenario in question involves moving optimizing the logistics of moving
  implements and supplies internally within a busy hospital.
  It consists of a central server in charge of planning and routing multiple
  robots. Additionally, it will be assumed planning will be done with
  OpenStreetMap and thus will use a graph-based approach. More details and
  specifics about this project will be discussed in a later section. \\

\end{document}
