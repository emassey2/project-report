%!TEX root = ../report.tex
\begin{document}
  \chapter{Criteria for Comparison}

  Given the wide variety and complexity of the methods available for
  spatio-temporal world modeling it's vital that criteria be established in
  order to properly and efficiently compare them. As seen in the previous
  section, some initial work has been done in this regard. This section,
  however, aims to introduce a complete list of criteria that can be used
  for comparison. The methods will be introduced, outlined, and their relevance
  to the field explained. At the end of the section is table that contains
  a number of methods for spatio-temporal world modeling as well as their
  respective information with regards to the criteria introduce in this section.
  The idea is to provide an overview of some common techniques for modeling
  and to establish a way to easily compare future spatio-temporal world modeling
  techniques.


  \section{ Criteria }

  \subsection{ Computational Complexity }
  According to Michael Stiff, a professor at the University of Wisconsin
  Madison, computational complexity can be defined as `` the study of how much
  of a given resource a program uses. The resource in question is usually
  either space (how much memory) or time (how many basic operations).''
  \cite{ComplexityUW} In the field of computer science this computational
  complexity is often represented using asymptotic notation, also commonly
  known as Big O notation. The general idea is to estimate how much of a given
  resource will be used as the size of the input data is increased, specifically
  as it approaches infinity. This is useful for estimating how much memory or
  how much time a given algorithm may use for any given set of data.
  Additionally, it serves an excellent benchmark for comparing various
  algorithms.\\

  TODO add picture from images of complexity growth rates

  In the particular case of spatio-temporal world modeling, one is most
  concerned about how long planning will take as well as how much memory the
  process will consume while planning. Furthermore, although Big O notation is
  still applicable and accurate within reason, some details may be lost.
  Specifically, Big O will often drop constants in growth rates as they are
  not relevant when looking at asymptotic behavior. An example of this with
  regard to spatio-temporal world modeling would be if one model stored a
  single copy of every observation made while another stored multiple, say
  five. Although the second model would grow five times as fast as the first,
  according to Big O notation, they would both be linear, thus growing at the
  same rate of N. This is obviously a draw back when the vast majority of
  modeling techniques would all be in a similar complexity class for both time
  and space. It is for this reasons when comparing spatio-temporal world
  models Big O notation may still be used, but coefficients will not be
  dropped in order to better convey the subtleties of the various methods.


  \subsection{ Efficiency of Data Storage }
  Similar to spatial and time complexity, the efficiency of storage also
  evaluates how a method handles the resources available to it. However, where
  spatial and time complexity are focused on the use of resources during
  run-time, the evaluation of the efficiency of storage is focused on how well
  the information used for predictions can be stored when not in use. This
  can be important when looking at the scaleability of a particular model. One
  can imagine if hundreds of thousands of observations are being made every
  day, depending on the data contained within those observations, an
  inefficient storage approach could rapidly cause giga- or terabytes of storage
  space to be used.

  \subsection{ Offline vs Online Learning }
  When discussing different models, it is important to distinguish when a model
  changes or updates in relation to when predictions are made. A common way to
  make this differentiation is by categorizing an algorithm or method into one of
  two categories: Online or Offline. Offline algorithms or methods are the
  historically common approach. Data is taken in an analyzed in batches before,
  in the case of spatio-temporal world modeling, predictions are made. Online
  learning on the other hand sees a much quicker turn around time between the
  observation of new data and the resulting prediction. \cite{Karp1992} For the sake of this
  paper, methods will be classified as online learning methods if they update
  after ever new observation or after a given number of observations. Methods
  will therefore be considered offline learning methods if they are designed to
  process large amounts of data before making predictions, specifically if it
  is not uncommon for the data processing time to take long enough as to make
  immediate and live predictions unreasonable.

  \subsection{ Map Type Dependency }
  As mentioned in the State of the Art section, certain spatio-temporal world
  modeling methods may have been developed with specific world modeling techniques in
  mind. It is quite conceivable that these different modeling methods would
  have an impact on the performance of prediction. Additionally, looking at it
  from a developer stand point, it may be desirable to extend a preexisting
  world model with the ability to make spatio-temporal predictions. It is
  therefore conceivable that an environment may already be modeled using a
  specific technique and thus limiting the pool of potentially available methods.
  For these reasons, it is desirable to be able to differentiate and thus choose
  between the various methods with relation to the world modeling method
  required. TODO this section might need a little love in terms of wording



  \subsection { Limitations on Predictions }
  When looking over a large number of options it is often helpful to eliminate
  choices that absolutely will not work. This section attempts to outline
  the type of limitations that could act as deal breakers for someone
  attempting to choose a spatio-temporal world model. TODO ReWoRdInG tHoUgH

  \subsubsection{ Classification Restrictions }
  Certain techniques used for predictions can cause limitations on the number
  or type of classifications that can be done on a set of data. The most common,
  and perhaps, most limiting, is the binary restriction imposed by many of the
  methods that use FreMEn TODO cite such as X, Y, Z. This restriction only
  allows an objects state in the spatio-temporal domain to take one of two
  states. This state is then said to have a probability associated with it.
  Thus, it is often not possible to represent more complex behavior like time
  of travel, number of people present in an area, or any other object or behavior
  that can take more than two states.

  \subsubsection{ Historical Predictability }
  The ability to reflect on and learn from previous mistakes is as useful in
  robotics as it is for humans. Being able to recreate and analyze historical
  events allows for not only debugging of unexpected behavior but also for
  improved training of a given model. These improvements can ultimate result
  in more accurate predictions. Therefore, the ability to predict events that
  have already happened can be a desired trait when looking for at
  spatio-temporal world modeling techniques.

  \subsubsection{ Long-Term Predictability }
  Conversely, compared to historical predictions, sometimes it may be desirable
  to make long-term future predictions about the world. That is to say, predictions
  not just about what is immediately going to happen, but about what may happen
  in the next few days, months, or even years. Many scenarios can be imagined
  that would benefit from this ability. Long-term future predictions enable
  increased autonomy as well as decreased requirements for offline learning.
  Furthermore, long-term predictions about the future can also be used to
  retroactively evaluate the performance of multiple models, perhaps using
  the same technique but with different parameters.


  \subsection{ Metainformation }

  Some information about a spatio-temporal world modeling technique may not be
  directly observable or even contained within the approach itself. Information
  about how and when a method is best used or other meta-information not directly
  tied to an approach, like how easy it is to setup, can be vital to a developer
  when attempting to decided between various methods. This section introduces
  some of the metainformation about spatio-temporal world modeling techniques
  that could prove useful, especially with regards to a developer or engineer.

  \subsubsection{ Availability of Work }
  Commonly, during or after the development of a new spatio-temporal world
  modeling technique the author or authors will release the new method in the
  form of a library or source code. Together, these two can save time developing
  and debugging and prevent extra work. Additionally, they may have an effect
  on what language will be used on a given project.

  \subsubsection{ Implementation Complexity }
  When a new technique is proposed but no implementation is available, it is
  up to a developer or engineer to implement the technique proposed in the paper.
  Depending on the complexity of the approach proposed, this can take anywhere
  from a day or two to complete to weeks to understand and implement. Depending
  on the requirements of a project, it may be desirable to avoid overly complex
  approaches that may consume limited development resources.

  \subsubsection{ Suitable Fields of Application }
  Although the field of spatio-temporal world modeling is generally concerned
  with how, when, and where things are happening, not all methods attempt to
  predict or model the same behavior. It is often common for a given modeling
  technique to be developed for a specific problem. Sometimes it is possible to
  modify these approaches to fit other fields. Other times, generic
  approaches are developed directly. This section aims to categorize how tied
  a method is to a given problem classification and well suited is to certain
  sets of problems. TODO reword maybe


  \section{ How to use this criteria }
  TODO discussion and table goes here


  \section{ TODO What do I do about this stuff? }

  TODO what about this stuff that needs to be done experimentally?
  \subsection{ Future Prediction Accuracy }
  \subsection{ Historical Recreation Accuracy }

  \subsection{ Feasibility of use with a multi-robot system }
  TODO what should I do about this? There isn't any research here




\end{document}
