%!TEX root = ../report.tex
\begin{document}
  \chapter{Criteria for Comparison}

  Given the wide variety and complexity of the methods available for
  spatio-temporal world modeling it's vital that criteria be established in
  order to properly and efficiently compare them. As seen in the previous
  section, some initial work has been done in this regard. This section,
  however, aims to introduce a complete list of criteria that can be used
  for comparison. The methods will be introduced, outlined, and their relevance
  to the field explained. At the end of the section is table that contains
  a number of methods for spatio-temporal world modeling as well as their
  respective information with regards to the criteria introduced in this section.
  The idea is to provide an overview of some common techniques for modeling
  and to establish a way to easily compare future spatio-temporal world modeling
  techniques.


  \section{ Criteria }

  \subsection{ Computational Complexity }
  According to Michael Stiff, a professor at the University of Wisconsin
  Madison, computational complexity can be defined as `` the study of how much
  of a given resource a program uses. The resource in question is usually
  either space (how much memory) or time (how many basic operations).''
  \cite{ComplexityUW} In the field of computer science this computational
  complexity is often represented using asymptotic notation, also commonly
  known as Big O notation. The general idea is to estimate how much of a given
  resource will be used as the size of the input data is increased, specifically
  as it approaches infinity. This is useful for estimating how much memory or
  how much time a given algorithm may use for any given set of data.
  Additionally, it serves as an excellent benchmark for comparing various
  algorithms.\\

  TODO add picture from images of complexity growth rates

  In the particular case of spatio-temporal world modeling, one is most
  concerned with how long planning will take and how much memory the
  process will consume while planning. Furthermore, although Big O notation is
  still applicable and accurate within reason, some details may be lost.
  Specifically, Big O will often drop constants in growth rates as they are
  not relevant when looking at asymptotic behavior. An example of this with
  regard to spatio-temporal world modeling would be if one model stored a
  single copy of every observation made while another stored multiple, say
  five. Although the second model would grow five times as fast as the first,
  according to Big O notation, they would both be linear, thus growing at the
  same rate of N. This is obviously a drawback when the vast majority of
  modeling techniques would all be of a similar complexity class for both time
  and space. It is for this reason when comparing spatio-temporal world
  models Big O notation may still be used, however in order to better convey
  the subtleties of the various methods coefficients will not be dropped.


  \subsection{ Efficiency of Data Storage }
  Similar to spatial and time complexity, the efficiency of storage also
  evaluates how a method handles the resources available to it. However, where
  spatial and time complexity are focused on the use of resources during
  run-time, the evaluation of the efficiency of storage is focused on how well
  the information used for predictions can be stored when not in use. This
  can be important when looking at the scaleability of a particular model. One
  can imagine if hundreds of thousands of observations are being made every
  day, depending on the data contained within those observations, an
  inefficient storage approach could rapidly cause giga- or terabytes of storage
  space to be required. Therefore, it is important to note once learned or trained,
  how and if information and predictions can be stored.

  \subsection{ Learning Method }
  The learning method is a way a spatio-temporal world model is able to
  make predictions. Often this is based off of preexisting concepts like neural
  nets or mathematical models. Information on these methods is commonly known
  and thus additional information may be desired to further compare methods.
  One such vital piece of information is distinguishing when a model
  changes or updates in relation to when predictions are made. A common way to
  make this differentiation is by categorizing an algorithm or method as one
  two categories: Online or Offline. Historically, offline algorithms or methods are the
  historically common approach. Data is taken in an analyzed in batches before,
  in the case of spatio-temporal world modeling, and predictions are made. Online
  learning on the other hand sees a much quicker turn-around time between the
  observation of new data and the resulting prediction. \cite{Karp1992} For the sake of this
  paper, methods will be classified as online learning methods if they update
  after ever new observation or after a given number of observations. Methods
  will therefore be classified as offline learning methods if they are designed to
  process large amounts of data before making predictions, specifically if it
  is not uncommon for the data processing time to take so long as to make
  immediate and live predictions unreasonable.

  \subsection{ Map Type Dependency }
  As mentioned in the State of the Art section, certain spatio-temporal world
  modeling methods have been developed with specific world modeling techniques in
  mind. It is quite probable that these different modeling methods would
  have an impact on the performance of prediction. Additionally,
  from a developer stand point, it may be desirable to extend a preexisting
  world model with the ability to make spatio-temporal predictions. It is
  therefore conceivable that an environment may already be modeled using a
  specific technique and thus limiting the pool of potentially available methods.
  For these reasons, being able to differentiate, and thus choose
  between the various methods with relation to the world modeling method
  required is extremely useful. TODO this section might need a little love in terms of wording



  \subsection { Limitations on Predictions }
  When examining a large number of options it is often helpful to eliminate
  choices that absolutely will not work. This section will attempt to outline
  the types of limitations would disqualify certain spatio-temporal world
  models when attempting to find a  best fit solution for a give problem.


  \subsubsection{ Classification Restrictions }
  Certain techniques used for predictions can cause limitations on the number
  or type of classifications that can be done on a given set of data. The most common
  and perhaps most limiting is the binary restriction imposed by many of the
  methods that use FreMEn TODO cite such as X, Y, Z. This restriction only
  allows an object's state in the spatio-temporal domain to be represented by one of two
  states. This state is then said to have a probability associated with it.
  Thus, it is often not possible to represent more complex behavior like time
  of travel, number of people present in an area, which can potentially be in
  more than two states.

  \subsubsection{ Historical Predictability }
  The ability to reflect on and learn from previous mistakes is as useful in
  robotics as it is for humans. Being able to recreate and analyze historical
  events allows for unexpected to be debugged and for
  improved training of a given model. These improvements can ultimately result
  in more accurate predictions. Therefore, the ability to predict events that
  have already happened can be a desired trait when looking at
  spatio-temporal world modeling techniques.

  \subsubsection{ Long-Term Predictability }
  Conversely, compared to historical predictions, sometimes it may be desirable
  to make long-term future predictions about the world. That is to say, predictions
  not just about what is immediately going to happen, but about what may happen
  in the short, near, or long term. Many scenarios can be imagined
  in which this would be advantageous. Long-term future predictions allow for
  increased autonomy and decreased requirements for offline learning.
  Furthermore, long-term predictions about the future can also be used to
  retroactively evaluate the performance of multiple models, perhaps using
  the same technique but with different parameters.


  \subsection{ Metainformation }

  Some information about a spatio-temporal world modeling technique may not be
  directly observable or even contained within the approach itself. Information
  about how and when a method is best used or other meta-information not directly
  tied to an approach, like how easy it is to set up, can be vital to a developer
  when attempting to decide between various methods. This section introduces
  some of the metainformation about spatio-temporal world modeling techniques
  that could prove useful, especially with regard to a developer or engineer.

  \subsubsection{ Availability of Work }
  Commonly, during or after the development of a new spatio-temporal world
  modeling technique the author or authors will release the new method in the
  form of a library or source code. The availability of libraries or source code
  can save time developing
  and debugging and prevent extra work. Additionally, the preexisting work may have an effect
  on what language will be used on a given project. The availability of work
  will be graded on scale of 1-3. 1 means that the work is
  freely available as a package or source code and may be open or closed source.
  2 means that some of the subcomponents may be freely available others,
  commonly the connecting or support structures are not yet implemented and it
  would take some time to connect everything together. A 3 means that there
  does not currently exist any available code for this approach. One would have
  to fully acquaint themselves with the original academic paper in order to
  implement this method.

  \subsubsection{ Implementation Complexity }
  When a new technique is proposed but no implementation is available it is
  up to a developer or engineer to implement the technique proposed in a given paper.
  Depending on the complexity of the approach proposed, this takes days or
  weeks to understand and fully implement. Depending
  on the requirements of a project, it may be desirable to avoid overly complex
  approaches that may consume limited development resources. For this reason,
  implementation complexity will be graded on a scale from 1-3 where 1 is a
  simple combination of preexisting math or models that a high
  level undergraduate student could understand and implement. 2 means a
  more complex system with multiple components of moderate complexity.
  3 is reserved for cutting edge implementations that almost always involve
  multiple complex components.

  \subsubsection{ Suitable Fields of Application }
  Although the field of spatio-temporal world modeling is generally concerned
  with how, when, and where things are happening, not all methods attempt to
  predict or model the same behavior. It is often common for a given modeling
  technique to be developed for a specific problem. Sometimes it is possible to
  modify these approaches to fit other fields. Other times, generic
  approaches are developed directly. This section aims to categorize how specific
  a method is for a given problem classification and how well suited is for certain
  sets of problems.


  \section{ Experimental Criteria }

  Comparing and contrasting various techniques on paper is an excellent way
  to narrow down one's options when choosing a modeling technique. However, at
  a certain point it is desirable to evaluate the remaining methods with
  respect to performance data. This information cannot always be obtained via
  research alone due to lack of information provided in the original or
  subsequent research. Additionally, results from  one experiment in a given domain
  may not directly translate to that of another. Therefore, in order to best
  compare and thus most effectively choose a technique for spatio-temporal
  world modeling, some experiments must be done. The following criteria are
  a recommended starting point for comparison, however, the specifics of a
  project may demand modified or additional analysis.

  \subsection{ Prediction Accuracy }
  Prediction accuracy is arguably the most important benchmark for for comparing methods
  of spatio-temporal world modeling. It can take many forms, but the goal is
  to convey on average how well a model can predict a future event. This is
  most commonly achieved by comparing a predictive model to the ground truth
  data. For binary data, this is often simply done by counting the number of
  incorrect predictions. Non-binary data becomes a bit more difficult and, depending
  on the data being worked on, there are multiple approaches. The easiest
  to convey is calculating and tracking the average distance from a correct
  result. Regardless of the data being tracked, the end result is best viewed
  in the form of a graph over time. This allows for the most concise comparison
  of the performance of various spatio-temporal world modeling techniques.

  \subsection{ Computational Resource Usage }
  Computational resource usage can be thought of as an extension of computational
  complexity into the real world. This statistic aims to quantify and measure
  the time, memory, or any additional resources a computer provides during run-time.
  Most commonly these resources are amount of time spent using a processor
  and the amount of memory used. These are commonly measured using seconds and
  bytes in a resolution scaled to the size of the problem,  milliseconds
  and kilobytes for example. These are excellent ways of predicting the ability
  of a given method to scale. In a real operating environment it does not matter
  if 100\% accuracy is obtained for all data sets if too much memory is required
  for a computer to process the results or if the time taken to
  produce the prediction invalidates the very prediction it makes.


  \subsection{ Feasibility of use with a multi-robot system }
  TODO what should I do about this? There isn't any research here


  \section{ How to use this criteria }
  TODO discussion and table goes here




\end{document}
