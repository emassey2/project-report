%!TEX root = ../report.tex

\begin{document}
    \chapter{Conclusions}

    \section{Contributions}

    First and foremost, this work introduced a comprehensive method for the
    comparison of various spatio-temporal world modeling methods. Although as
    mentioned in the State of the Art section some previous work in this area
    already exists, it was limited both in its ability to be applied to various
    methods as well as the depth at which any two methods could be compared. The
    new techniques for comparison established in the Criteria for Comparison
    section build on the state of the art increasing its breadth and depth.
    Furthermore, the ROPOD case study section shows in detail how these criteria
    can be used with respect to an actual project. This is an extremely important
    contribution as all previous work merely used criteria to look reflexively
    at the quality of a new approach and not instead at which method would be
    best suited for a given problem. \\

    Building on top of the qualitative analysis work already presented in this
    paper, the experimental section provided insight into how one would select
    a set of experiments to test and refine the previously obtained results.
    These experiments highlight unique aspects of the techniques presented that
    were not necessarily highlighted in their original papers. Additionally,
    this section provides a road map into what type of experiments should be
    done, what results to look for, and how to evaluate those results. \\

    Finally, for both the qualitative and experimental sections, a huge major
    was placed on work being done with respect to a real-world project. In the
    past although work has been done with real world projects, the comparative
    techniques have been primarily focused on the performance of the algorithm
    itself. That is to say previous work has focused solely on how ``correct''
    a spatio-temporal modeling technique was for an individual object. This work
    focuses on additional information like computational intensity, as well
    as performance when working with the resulting predictions. \\

    \subsection{ Recommendations for ROPOD }

    \section{ Lessons Learned }

    \section{ Future work }

    Although this work has made noticeable contributions to the field, it is
    certain that much work remains. Once again, ROPOD can be used as an example
    to base future work around. ROPOD, unlike previous work, is a multi-robot
    system. This means that while running test with respect to a single robot
    is valid and fully accurate the addition of multiple robots creates
    additional opportunities and complexities. With multiple robots moving about
    the hospital at different times there is the possibility of gathering much
    more data about the environment which decreases the sparsity of data
    collected. This is theorized to increase model accuracy, as visible in the
    elevator experiments, but may also cause issues when trying to deal with
    or merge multiple samples of the same object at the same time. Moreover,
    the prescience of multiple robots may themselves introduce spatio-temporal
    complexities themselves. Having multiple robots moving about a shared
    space, especially during peak traffic, could create additional obstacles
    in the form of congestion that would need to be planned around. \\

    Another area of future work as bravely touched on above is the collection
    of additional real-world data. This data is often more noisy, irregular,
    and generally unpredictable when compared to even the best simulated data
    It would be ideal if this new data could be collected and compiled into a
    new dataset. Early steps have been been taken as exemplified in the
    'Brayford' dataset \cite{Krajnik2014}. This dataset, however, contains
    3D data for a single room. Observation was done using a static device and
    thus observations are at regular intervals and contain minimal noise.
    Furthermore, because this is was only one device the complexities of a
    multi-robot system are also overlooked. All of this combines to make a
    rich opportunity ripe for exploration.

\end{document}
