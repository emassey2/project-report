%!TEX root = ../report.tex

\begin{document}
    \begin{abstract}
    %The improvement of a robot's world model often directly correlates with
    %improved path planning and motion performance. Spatio-temporal world
    %models are a relatively new and innovative way of improving these world models
    %by incorporating information about how a given map changes relative to
    %time. The field of logistics, the coordination of materials from one point to
    %another, is a very common task in robotics that relies heavily on
    %up-to-date information, especially with respect to a robot's world model.
    %The development of spatio-temporal world models is therefore of great
    %interest to the field of logistics, as improving path planning in a
    %supply chain increases throughput, and saves time and money. \\

    Currently, a variety of methods exist for creating different
    types of spatio-temporal world models. Despite the numerous methods for
    this type of modeling, there exists no methodology for comparing the
    different approaches or their suitability for a given application e.g. logistics robots.
    In order to establish a means for comparing and selecting the best-fitting
    spatio-temporal world modeling technique, a methodology and standard set
    of criteria must be established.
    To that end, state-of-the-art methods for this type of
    modeling will be collected, listed, and described. Existing methods used
    for evaluation will also be collected where possible. Using the collected
    methods, new criteria and techniques will be devised to enable the
    comparison of various methods in a qualitative manner.
    Experiments will be proposed to further narrow and ultimately select a
    spatio-temporal model for a given purpose. An example
    network of autonomous logistic robots, ROPOD, will serve as a case study used to demonstrate
    the use of the new criteria. This will also serve to guide the design of
    future experiments that aim to select a spatio-temporal world modeling technique for a
    given task. ROPOD
    was specifically selected as it operates in a real-world, human shared
    environment. This type of environment is desirable for experiments as it provides a unique
    combination of common and novel problems that arise when selecting an
    appropriate spatio-temporal world model.
    Using the developed criteria, a qualitative analysis will be applied to the
    selected methods to remove unfit options. Then, experiments will be run on
    the remaining methods to provide comparative benchmarks.
    Finally, the results will be
    analyzed and recommendations to ROPOD will be made.
\end{abstract}
\end{document}
