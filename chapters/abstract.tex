%!TEX root = ../report.tex

\begin{document}
    \begin{abstract}
    The improvement of a robots world model often directly correlates with
    improved path planning and motion performance. Spatio-temporal world
    models are a relatively new and innovative way of improving world models
    by incorporating information about how a given map changes relative to
    time. The field of logistics, that is moving objects from one point to
    another, is a very common task in robotics that relies heavily on
    up-to-date information especially with respect to a robots world model.
    The application of spatio-temporal world models is therefor of a great
    interest to the field of logistics, as improving path planning in a
    logistics network increase throughput, and saves time as well as money.
    Currently there exist a variety of methods for creating many different
    types of spatio-temporal world models. Despite the numerous methods for
    this type of modeling, there exists no methodology for comparing the
    different approaches or their suitability for a given logistics network.
    In order to enable the comparison and selection of a fitting
    spatio-temporal world modeling technique criteria and methodology must be
    established. To this end, state of the art methods for this type of
    modeling will be collected, listed, and described. Existing methods used
    for evaluation will also be collected where possible. Using the collected
    methods, new criteria and techniques will be devised to enable the
    comparison of the various methods in a qualitative manor. Experiments will
    also be proposed as a means of further narrowing and ultimately selecting
    a spatio-temporal model for a given purpose. An example autonomous
    logistic network, ROPOD, will be used as a case study demonstrating the
    use of the newly established criteria and to enable to design of
    experiments that may need to be run to select a modeling technique. ROPOD
    was specifically selected as it operates in a real-world, human shared
    environment. This type of environment is desired as it provides a unique
    combination of common and novel problems when attempting to select an
    appropriate spatio-temporal world model. A qualitative analysis using the
    proposed criteria will be applied to the state of the art methods
    collected resulting in a narrow field of candidates. Experiments will be
    run to benchmark the remaining methods. Lastly, the results will be
    analyzed and recommendations to ROPOD will be made.
\end{abstract}
\end{document}
