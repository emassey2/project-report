%!TEX root = ../report.tex

\begin{document}
  \chapter{ROPOD: A Case Study}

  With a large number of methods for spatio-temporal world modeling outlined
  in the State of the Art section, this section will apply the comparison tools
  to an example case study. For the
  purposes of this paper the ROPOD project has been selected as the
  candidate for the case study. \\

  \section{ What is ROPOD? }

  ROPOD is a research project
  funded by the ``European Union's Horizon 2020 research and innovation programme''
  that aims to develop ``ultra-flat, ultra-flexible,
  cost-effective robotic pods for handling legacy in logistics``
  \footnote[1]{https://www.h-brs.de/en/ropod}. In additional to its business
  oriented goals, ROPOD, being a research project, also has a focus on the
  development and implementation of cutting edge technology. The main goal of
  the project is to enable low cost robotic solutions for coordinating logistics
  in a preexisting human environment. Specifically, the current target
  testbed for ROPOD is a mid sized hospital in Frankfurt, Germany.
  ROPOD is comprised of a multi-robot system combined with a central server.
  Although the individual robots are in charge of navigation and orientation
  on a local level, the central server is responsible for planning when and
  how specific actions are carried out. In particular, the central server uses a
  modified and extended version of
  OpenStreetMap\footnote[2]{https://www.openstreetmap.org}, also known as OSM, to
  maintain the main global world model.
  This means that all information storage and path planning needs to work in
  conjuction with OSM. Of specific interest to this project is the global path
  planning, which focuses on modeling both individual objects in an
  environment and maintaining a graph with nodes and edges. \\


  \section{ Why ROPOD? }
  ROPOD offers both benefits and interesting technical
  challenges that make it excellent for a case study. First and
  foremost, one of the main benefits is that the robots involved will be
  operating in a real-world environment with all the intricacies and
  complexities that accompany it. Of particular note for this case study is
  that the operating environment is cohabited by humans. It is
  postulated that many of the behaviors and dynamic obstacles encountered
  will be of periodic and thus predictable nature. Furthermore, because ROPOD
  is a multi-robot system, as the number of robots increases, so will the
  frequency and range of data collected. It is believed that this
  will further increase accuracy of predictions, or at the very least increase
  the size and scope of the dataset of training. Because the targeted
  area of operation is a static logistics network, this allows for paths to be
  planned a significant time in advance. Furthermore, the use of a central
  server allows for more complex calculations and predictions to be done.
  In total, these benefits allow a both a methodical and comprehensive model
  to be developed. \\

  However, ROPOD is not without its technical complexities. A
  result of the operating environment being both dynamic and human, many
  different types of objects and behaviors must be modeled. Binary objects
  such as doors being open or closed, are obvious examples, but more
  complex, non-binary objects must also be efficiently and accurately modeled.
  Complex variables in this case range from the wait time after an elevator has been called,
  to the average travel time between two nodes, and the
  density of people in a specific area. All of these behaviors and more must be
  modeled for any given time and location in the hospital. This provides considerable
  test space with numerous areas for technical complications. Additionally,
  although ROPOD is being funded and targeted for research, it still requires
  efficient results and forces projects to focus on practical results within
  a reasonable time frame. This means that even if an algorithm or method
  is theoretical the best on paper, it may not be the best solution for ROPOD
  if it consumes to many computational resources or has other requirements that
  cannot be met. To that effect ROPOD must integrate
  with existing technology that has already been integrated with the ROPOD system. This
  means that any spatio-temporal world modeling method chosen must mesh well with OSM and other design
  elements. \\



  \section{ Methods Under Consideration }

  Drawing from the spatio-temporal world modeling methods mentioned in the
  State of the Art section, a handful have been selected for discussion here.
  The methods selected for additional analysis here were chosen specifically
  to allow for a variety of modeling techniques to be compared while also
  allowing for comparison of some similar styles in order to highlight the
  benefits of a comparative evaluation. They provide both varied range of
  methods and multiple methods of similar styles like have been included for
  further in-depth comparisons. \\

  Therefore, the following methods will be evaluated for use with ROPOD:

  \begin{itemize}
    \item Occupancy Grids with Input-Output Hidden Markov Model (IOHMM) - Modeling Spatial-Temporal Dynamics of Human Movements for Predicting Future Trajectories \cite{Wang2015}
    \item Conditional Transition Maps (CTM) - Conditional Transition Maps: Learning Motion Patterns in Dynamic Environments \cite{Kucner2013}
    \item Spatio-Temporal Hilbert Maps (STHM) - Spatio-Temporal Hilbert Maps for Continuous Occupancy Representation in Dynamic Environments \cite{Senanayake2016}
    \item Dynamic Multi-Map (DMM) - Dynamic Maps for Long-Term Operation of Mobile Service Robots \cite{Biber2005}
    \item Frequency Map Enhancement (FreMEn) - FreMEn: Frequency Map Enhancement for Long-Term Mobile Robot Autonomy in Changing Environments \cite{Krajnik2015}
    \item Hypertime - Warped Hypertime Representations for Long-term Autonomy of Mobile Robots \cite{Krajnik2018}
  \end{itemize}

  The following sections contain a brief overview of the selected methods.
  Although they will not cover all the specifics of a given model, they will
  mention the relevant information for comparison. Sources have been provided
  for each paper if the reader wishes to build a deeper understanding of one
  or more approaches. \\

  TODO review complexities per Dylan

  \subsection { Occupancy Grids with Input-Output Hidden Markov Model (IOHMM) }
  As the name implies, this method is primarily inspired by, and meant to be
  used with, occupancy grids. This effects many of the design decisions and
  resulting performance metrics. Aside from the regular computational complexity
  associated with neural networks, the Markov model was designed to be
  used with 4+1 edges. One for each of the neighbors, up, down, left, and right,
  as well as one for the center, or no change. Although this is how it was first
  described in the paper, it is possible to imagine this being extended to any
  arbitrary number of connections for use with other grids or graphs with more
  or less than 4 neighbors per cell or node. Despite this possibility, as it is
  described in the paper, the method is specifically designed with HMM occupancy
  grids in mind and thus is considered map dependent. Since the HMM weights can be
  be saved the data storage size reasonably low. However, because an HMM is used
  and only one model is stored there are no future or historical predictions
  only current predictions are possible. Due to the use of neural nets
  the learning method is considered offline because significant training must be
  done for prediction. Finally, although no source code or package is directly
  available, this method relies heavily on preexisting technology like HMM,
  occupancy grids, and neural nets. Using the paper as a guide it to keep
  reasonable to tie these technologies together without too many difficulties,
  not withstanding the standard roadblocks when working on a new project. \\

  \subsection { Conditional Transition Maps (CTM) }
  CTMs, as proposed in ``Conditional Transition Maps: Learning Motion Patterns
  in Dynamic Environments'' \cite{Kucner2013}, are similar to IOHMM on a conceptual
  level. They both involve using the connections between occupancy grid cells
  to model and predict future behavior. For similar reasons they are unable to
  do historical or long-term future predictions and are restricted to only
  predicting tendency of motion over time. The implementation details between
  the two are, however, different. CTM has additional connections at each cell.
  There connections for the four cardinal directions from a cell
  along with intermediates which increases the number of connections
  from 4 to 8. Furthermore, the direction of travel, in or out, is also
  considered. Combined, this means that there are 64 parameters per cell, one
  parameter for each combination of input and output direction. Neural nets
  are also not used for learning in this and instead a custom designed
  method is proposed. Due to the use of a custom method for prediction, and
  the lack of work available the implementation complexity is increased to a 3. \\

  \subsection { Dynamic Multi-Map (DMM) }
  The dynamic multi-map approach heavily depends on the number of maps used and their
  respective timescales. Although DMM was originally intended for use with
  occupancy grids. It can relatively easily be implemented to represent any
  number of objects or behavior. Additionally, selection of the maps is not
  defined. The maps themselves being a series of averages means this method acts as
  an online learning method. Because of this, historical and future predictions
  are not inherently possible. If maps were stored then historical predictions
  could be possible but this would also drastically decrease the storage
  efficiency dependent on the rate of sampling. Finally, although no current
  implementation of DMM is publicly available, given the relative ease of
  maintaining multiple maps and keeping a given amount of data during updates,
  DMM is fairly simple to implement by hand. The major outstanding drawback
  is the lack of a well defined method for choosing between or combining maps.
  This oversight is unfortunate as the method for selection or merging the
  map/maps has a large impact performance metrics of DMM. \\

  \subsection{ Spatio-Temporal Hilbert Maps (STHM) }
  STHMs extend existing Hilbert maps with the goal of providing real-time
  dynamic object tracking and prediction. Objects are detected as moving when
  two or more scans show motion between their estimated centers. These objects
  are then given motion vectors at every given time t. This means that not
  only does model complexity scale with the number of dynamic objects, it also
  scales with desired length of historical observations being made.
  Historical predictions are possible due to the
  fact that observations are being kept over time and predictions on objects
  are also being made over the same time.
  Furthermore, future predictions can be made with included uncertainty as
  time increases. This method uses many preexisting technologies, such
  as Hilbert maps, stochastic gradient descent, and Gaussian process regression.
  Considering the large and diverse number of techniques and the complexity with
  which they are connected to make a working system, this approach is considered
  to have an extremely high implementation complexity. Finally, no existing
  approach is currently available for use. \\

  \subsection { Frequency Map Enhancement (FreMEn) }
  FreMEn relies heavily on the use of a modified version of the Fourier
  transform and this decision informs a lot of it's properties. A model order must
  be defined for FreMEn. This order, which can be thought of as the number of
  cosines with coefficients that will be summed to make a prediction, dictates
  both time and, to some degree space complexity for training. Once trained,
  however, predictions can be obtained with a single, relatively computational
  inexpensive, equation. Therefore, FreMEn can be
  viewed as an offline learning method where training is done over a relatively
  long period of time and a model can be used for a given set of time before
  training again. Additionally, because predictions are done by computing a
  result at a given time t, predictions can be made at any arbitrary time,
  past, present, or future. FreMEn is well suited for doing predictions on any type
  of map or object with the caveat that the data must be binary, that is
  data can only be classified into one of two categories. Finally, since the
  implementation heavily relies upon a commonly understood concept with only
  minor support, it is considered a 2 for complexity. This is non-issue,
  however, as FreMEn is available both on
  GitHub \footnote[0]{https://github.com/strands-project/fremen}
  as well as in packages available for Ubuntu that add ROS
  support. \footnote[1]{http://strands.acin.tuwien.ac.at/software.html}


  \subsection { Hypertime }

  Created as an extension to FreMEn, Hypertime does not have a binary restriction
  on the classification of data. Additionally, data is now dynamically analyzed
  and wrapped back in on itself to improve observation coverage and improve
  predictions. Both of these behaviors are heavily dependent on the clustering
  method, of which Hypertime currently provides two; K-means clustering and
  Expectation Maximization. Both of these methods are discussed at length in
  the original paper, but their main goal is determining where and when the data observed
  should be looped back in on itself. Depending on the order provided and the
  number of clusters estimated, a linear scaling performance hit is expected.
  Once again, because Hypertime uses FreMEn, historical and future predictions
  are possible. Additionally, this work is openly available on GitHub
  \footnote[2]{https://github.com/gestom/Hypertime-RAL-18-0278/tree/master/door\_state/src/models}
  thus making the work availability a 1. On the other hand, due to the added
  complexity of wrapping space time and the inclusion of additional classification
  components, complexity has been bumped up to a 3. \\

  \subsection{ Summary of Approaches }

  The information gathered in the previous sections has been summarized in the
  following table:

  TODO fix this table somehow please

  % Please add the following required packages to your document preamble:
  % \usepackage{graphicx}
  %\begin{table}[]
  %\resizebox{\textwidth}{!}{%
  %Name
  %Additional Model Complexity
  %Learning Method
  %Offline vs Online
  %Efficiency of Data Storage
  %Map Dependent
  %Restriction(s)
  %Historical Predictions
  %Long-Term Predictions
  %Work Availability
  %Implementation Complexity
  %Suitable Fields of Application
  %\begin{tabular}{|p{2cm}|p{4cm}|p{3cm}|p{2cm}|p{3cm}|p{2cm}|p{3cm}|p{2cm}|p{2cm}|p{2cm}|p{3cm}|p{3cm}|}
  \begin{center}
  \rotatebox{90}{
    \scalebox{0.6}{
      %\begin{tabular}{|l|l|l|l|l|l|l|l|l|l|l|l|}
        \begin{tabular}{|p{2.3cm}|p{4cm}|p{3cm}|p{2cm}|p{3cm}|p{1.5cm}|p{3cm}|p{2cm}|p{2cm}|p{2cm}|p{3cm}|p{3cm}|}
        \hline
        Name      & Additional Model Complexity                                                & Learning Method             & Offline vs Online & Efficiency of Data Storage                     & Map Dependent & Restriction(s)         & Historical Predictions & Long-Term Predictions & Work Availability & Implementation Complexity & Suitable Fields of Application                                     \\ \hline
        IOHMM     & (4+1)*Cells                                                                & Neural Net                  & Offline           & HMM weights                                    & Yes           & Only motion prediction & No                     & No                    & 2                 & 2                         & Motion prediction                                                  \\ \hline
        CTM       & (8*8)*Cells                                                                & Cross-correlation           & Offline           & Cell Connections                               & Yes           & Only motion prediction & No                     & No                    & 3                 & 3                         & Motion prediction                                                  \\ \hline
        DMM       & Scales linear with number of maps                                          & Stochastic Sampling         & Online            & Varies with number of maps and samples per map & No            & N/A                    & No                     & No                    & 3                 & 1                         & Very versatile. Works with both binary and continuous observations \\ \hline
        STHM      & Scales linear with number of dynamic objects being tracked                 & Stochastic Gradient Descent & Online            & 2D vector per object at every time t           & Yes           & N/A                    & Yes                    & Yes                   & 3                 & 3                         & Live dynamic object tracking and prediction                        \\ \hline
        FreMEn    & Scales linear with Fourier Order                                           & Modified Fourier Transform  & Online            & Order dependent summation of cosines           & No            & Binary Predictions     & Yes                    & Yes                   & 1                 & 2                         & Binary data prediction (e.g. doors)                                \\ \hline
        Hypertime & Scales linear with Fourier Order \& further dependent on clustering method & FreMen + time folding       & Online            & Order dependent summation of cosines           & No            & N/A                    & Yes                    & Yes                   & 1                 & 3                         & Very versatile. Works with both binary and continuous observations \\ \hline
        \end{tabular}%
        %\caption{Summary of methods under consideration}
        %\label{figure:methods_summary}
      }
  }
  \end{center}

  \section{ Method Elimination }

  After collecting a number of different potential methods for spatio-temporal world
  modeling, a project must then reduce the number of methods to a smaller, more
  manageable amount. The methodology for selecting methods to eliminate will
  vary between projects, but most often it is useful to identify properties
  of the desired method that are categorically missing from one or more other
  methods. In the case of ROPOD, it is important that the method be able to
  represent multiple dynamic objects, represent binary and non-binary states,
  and be either readily available for use in the form of a library or reasonable to
  implement in order to match the strict work plan timeline. Additionally, although
  things like computational complexity, learning method, and offline vs online
  learning are important, they will not be the main focus of this section.
  Although, these categories give valuable insight into the potential performance
  of a method, at least in the case of ROPOD,
  it is assumed that there is a bit more leniency in computational performance
  due to the use of a central server.
  Lastly, although a single issue with a method may be problematic it should
  not immediately disqualify it as there is likely to be a work around. Instead,
  any issues with a method should be evaluated to asses if the effort needed
  to resolve the issue, or the impact the work around will have, is enough to
  justify that methods use. This is especially important to evaluate when
  multiple issues with a method arise. In this case it becomes increasingly
  unlikely that a method will be feasible and thus can be eliminated.
  With this in mind, the above methods will be
  evaluated one by one resulting in a pared down list of methods that can
  then be tested. \\

  \subsection { Occupancy Grids with Input-Output Hidden Markov Model (IOHMM) }
  IOHMM was originally designed for and preforms best when doing motion prediction.
  Unfortunately, that poses an issue for this case study. Although motion prediction
  may be valuable for ROPOD in other scenarios, it is not applicable to this specific case study
  which places a focus on other areas of prediction.
  Another major issue this method has is its
  dependency on using an occupancy grid. Similar to the issue with motion
  prediction project, this specific case study uses a graph-like structure for path
  planning so although it may not be an issue in general, it is a problem for ROPOD. This method would also not be able to represent individual objects,
  yet another issue. Finally, with mediocre, availability and
  implementation complexity, it is clear that IOHMM is not a good fit for this
  case study. \\

  \subsection { Conditional Transition Maps (CTM) }
  CTM follows in the footsteps of IOHMM. Its dependency on occupancy grids as
  and the fact that it is best suited for motion prediction are major drawbacks.
  Additionally, the combination of work availability, or lack thereof, and
  its increased implementation complexity make it the easiest option to
  eliminate from this list. \\

  \subsection { Dynamic Multi-Map (DMM) }
  DMM, is an interesting
  method with respect to this case study. Despite its somewhat naive approach
  of averages, it does look suitable to the case study on paper. Importantly, it is map
  independent, allowing for the representation and prediction of both graph edges and individual
  objects. Additionally, it has no prediction restrictions on the type of
  behaviors or objects it can make predictions about. However, it has no known publicly available
  work which would be a drawback, but given its design simplicity, should
  be relatively easy to implement. DMM's lack of future or historical
  predictions is a bit of an inconvenience, but this can be overlooked as it is not
  strictly necessary for ROPOD. Given these reasons it will not be eliminated
  during the qualitative analysis. \\

  \subsection{ Spatio-Temporal Hilbert Maps (STHM) }
  STHM is another very interesting method with respect to ROPOD.
  It's live dynamic object tracking and prediction, while useful for some
  cases of ROPOD, does not meet all of the evaluation requirements.
  Much like CTM and
  STHM, is map dependent which is another drawback as the maps for
  ROPOD have already been chosen and defined. Unlike CTM and STHM, it does not
  have the motion-only prediction restriction which does make it a bit more
  desirable. Unfortunately, the final eliminating factors for STHM are the lack
  of work availability and high implementation complexity. It is possible that with some
  work that these issues could be overcome, but solving these issues would take
  a considerable amount of time. Combined with the time needed to write
  and implement the method from scratch, STHM is not suitable for this
  use case. \\

  \subsection { Frequency Map Enhancement (FreMEn) }
  FreMEn acts somewhat like a foil to STHM, at least with respect to it's
  qualitative analysis and possible elimination. Its map independent which is
  a quick mark in its favor. Its very versatile and could represent both graph
  edges as well as individual objects in an environment. One major restriction
  and mark against it though is its inability to represent objects or behaviors
  that are non-binary, that is to say, objects or behaviors that have more than
  two states. Despite this restriction, the work is readily available which
  nudges it just on the other side of elimination. Time will have to be spent
  to work around this binary restriction, but ultimately, FreMEn may be a good
  solution to this case study. At the very least, it could fit well for
  representing if objects like doors and hallways are passible at a given time.
  Lastly, the ability to make historical or long-term future predictions is
  a minor, but appreciated plus. \\

  \subsection { Hypertime }
  Hypertime is perhaps the best-looking method on paper. Because it is an
  extension of FreMEn it's map independent
  and has historical and future predictions capabilities. Unlike FreMEn, and to its benefit, it does not have
  restrictions on the type of predictions it can make. This is
  a huge advantage. Finally, with preexisting work being readily available, Hypertime
  is an excellent contender and will be tested in the experimental section. \\

  \subsection{ Results }
  \subsubsection{ Eliminated Methods }
  The following methods have been eliminated:
  \begin{itemize}
    \item Occupancy Grids with Input-Output Hidden Markov Model (IOHMM) - Modeling Spatial-Temporal Dynamics of Human Movements for Predicting Future Trajectories \cite{Wang2015}
    \item Conditional Transition Maps (CTM) - Conditional Transition Maps: Learning Motion Patterns in Dynamic Environments \cite{Kucner2013}
    \item Spatio-Temporal Hilbert Maps (STHM) - Spatio-Temporal Hilbert Maps for Continuous Occupancy Representation in Dynamic Environments \cite{Senanayake2016}
  \end{itemize}

  \subsubsection{ Methods Selected for Experimental Testing }
  The following methods will be tested further in the experimental section:
  \begin{itemize}
    \item Dynamic Multi-Map (DMM) - Dynamic Maps for Long-Term Operation of Mobile Service Robots \cite{Biber2005}
    \item Frequency Map Enhancement (FreMEn) - FreMEn: Frequency Map Enhancement for Long-Term Mobile Robot Autonomy in Changing Environments \cite{Krajnik2015}
    \item Hypertime - Warped Hypertime Representations for Long-term Autonomy of Mobile Robots \cite{Krajnik2018}
  \end{itemize}

\end{document}
